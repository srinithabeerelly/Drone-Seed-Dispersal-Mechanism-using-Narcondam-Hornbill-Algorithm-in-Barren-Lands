\documentclass[pmlr,twocolumn,10pt]{jmlr} 
\usepackage{graphicx} % Required for inserting images

\title{Narcondam Hornbill algorithm
\\
BEERELLY SRINITHA and SIKANDER KATHAT}
\author{ }
\date{February 2023}

\begin{document}

\maketitle
\begin{center}
Narcondam Hornbill Algorithm -
An Algorithm for Automated and efficient Seed Dispersal in Barren Lands
\end{center}
\section*{Abstract}
The degradation of fertile land due to environmental factors such as desertification, overgrazing, and deforestation has resulted in the creation of vast areas of barren land. In order to combat this, it is necessary to develop new methods for the rejuvenation of such areas. In this paper, we present a new algorithm for automated seed dispersal in barren lands. Seed dispersal by drones is an innovative approach to restore vegetation in barren lands, inspired by the Narcondam hornbill, a bird found in the Andaman and Nicobar Islands. 
\\This paper outlines the potential steps for implementing seed dispersal by drones, starting with identifying suitable plant species for the target area, collecting seeds, and preparing seed mixtures. The drone design and flight parameters such as altitude, speed, and timing are critical for successful seed dispersal. The algorithm takes into account various factors such as climate, topography, soil type, and seed variety to determine the optimal number of seeds per unit area, and dispatches a drone to the specified location to begin seed dispersal. The algorithm also takes into account the location to determine the optimal seed dispersal rate.
\\The paper concludes that seed dispersal by drones has great potential to restore vegetation cover in barren lands, but more research is needed to optimize the approach and ensure its long-term success.



\section*{Introduction}
Seed dispersal is a crucial process that contributes to the survival and diversity of plant species. In nature, seed dispersal is carried out by a variety of mechanisms such as wind, water, animals, and birds. Among these, avian seed dispersal, or ornithochory, is considered to be one of the most effective methods for seed distribution, as birds can cover large areas quickly and disperse seeds over a range of habitats. In recent years, computer scientists have developed various algorithms for simulating seed dispersal and optimizing the process for real-world applications. However, many of these algorithms are limited in their efficiency and accuracy and do not fully capture the complexity of the natural processes of seed dispersal.
\\In this report, we present a new seed dispersal algorithm inspired by the endozoochory process of the Narcondam Hornbill. The Narcondam Hornbill is a bird found only on the remote island of Narcondam in the Andaman and Nicobar archipelago and plays a critical role in the seed dispersal of many plant species. By consuming fruits and seeds, the bird helps to transport them to new locations, where they can germinate and grow. Our algorithm is designed to simulate the endozoochory process of the Narcondam Hornbill and to optimize seed dispersal for real-world applications. We discuss the various components of our algorithm, including the selection of suitable plant species, the design of the seed mixture, and the simulation of bird movement patterns. We also evaluate the efficiency and accuracy of our algorithm and compare it with existing seed dispersal algorithms.
\\Overall, our research suggests that the Narcondam Hornbill can serve as an important source of inspiration for developing more effective seed dispersal algorithms, and highlights the potential of avian seed dispersal for restoring and maintaining biodiversity in natural ecosystems. Our study contributes to the growing body of knowledge on the role of birds in ecosystem functioning and emphasizes the need for innovative and sustainable solutions to address the global challenge of seed dispersal in barren lands.

\section*{Background}
The Narcondam Hornbill (Rhyticeros narcondami) is a bird species found only on the small island of Narcondam in the Andaman and Nicobar archipelago of India. It plays an important role in seed dispersal on the island.
\\The Narcondam Hornbill feeds on a variety of fruits, including figs, guavas, and other fleshy fruits. As the bird consumes the fruit, the seeds pass through its digestive system and are deposited in its droppings in different locations across the island.
\\This process of seed dispersal through ingestion by birds is called endozoochory. The seeds in the hornbill's droppings have a better chance of germination and growth because they have been partially digested and their hard outer shells have been broken down, making it easier for the young plant to emerge.
\\Overall, the Narcondam Hornbill is an important part of the island's ecosystem, helping to disperse the seeds of many plant species and contributing to the island's biodiversity.
\\However, the Narcondam Hornbill is currently classified as a vulnerable species due to habitat loss and fragmentation, making it increasingly important to study and understand its role in seed dispersal.
\\Furthermore, seed dispersal is a crucial process for maintaining and enhancing biodiversity in natural ecosystems. It allows for plant species to colonize new areas, increase genetic diversity, and support the survival of other organisms. Therefore, developing effective seed dispersal algorithms that can simulate and optimize the process is important for restoration and conservation efforts.
\\Several existing seed dispersal algorithms have been developed, but many are limited in their accuracy and efficiency. As a result, there is a need to explore and develop new algorithms inspired by the natural processes of seed dispersal. The Narcondam Hornbill, with its unique role in endozoochory, provides a valuable source of inspiration for such algorithms.

\section*{Methods}
 This algorithm takes in a list of images which are captured by the drone during the aerial survey as input, stitches them together into a single image, converts the image to the HSV color space, applies a color threshold to identify pixels within the range of barren land color, and then calculates the total area of the barren land in the image and returns the area of the barren land as output.
 \\The algorithm then starts checking if the climate conditions are suitable for the selected seed variety. If the climate is suitable, the algorithm returns a message indicating that the climate is suitable for the selected seed variety. The algorithm also checks if the soil type is appropriate for the selected seed variety. If the soil type is suitable, the algorithm returns a message indicating that the soil type is suitable for the selected seed variety.
\\Next, the algorithm takes in the climate conditions, topography, soil type, and seed variety as input looks up the recommended seed density for the given seed variety based on the current growing conditions, and returns the recommended seed density as the optimal number of seeds per unit area. 
\\The algorithm then takes in the location where the seeds need to be dispersed and the total number of seeds as input, determines the number of seeds to place in each row and column, determines the spacing between each seed, creates a seed object, and returns a list of tuples containing the location of each seed along with its associated seed object. 
\\The algorithm then takes in the soil type as input, creates a seed object with the necessary properties such as water level, nutrient level, etc, and returns the seed object. The algorithm takes in the topography and location as input, determines the optimal seed dispersal rate of seeds based on the topography and location, and returns the dispersal rate. The drone is then dispatched to the location and begins seed dispersal. If the seed dispersal is successful, the algorithm returns a message indicating that the seed dispersal was successful. If the seed dispersal is not successful, the algorithm returns a message indicating that the seed dispersal failed.

\section*{Algorithm in brief}

\\
\\
\\
\\
\section*{Schematic work and flowcharts}

\section*{Future work}


\section*{Conclusion}
In conclusion, the algorithm presented in this paper offers a new and automated solution for seed dispersal in barren lands. The algorithm takes into account various factors such as climate, topography, soil type, and seed variety to determine the optimal number of seeds per unit area, and dispatches a drone to the specified location to begin seed dispersal. The algorithm also takes into account the topography and location to determine the optimal seed dispersal rate. The algorithm has the potential to greatly improve the efficiency and effectiveness of seed dispersal in barren lands and may help to mitigate the effects of environmental degradation.The Narcondam Hornbill-inspired algorithm offers a more efficient and accurate way of simulating seed dispersal in a simulated environment. The algorithm's simulation of the hornbill's dispersal process and its consideration of environmental factors improve the accuracy and efficiency of the seed dispersal process. In the future, the algorithm could be further refined and tested in real-world applications such as reforestation and conservation efforts.

In conclusion, the Narcondam Hornbill-inspired seed dispersal algorithm provides a novel approach to improving seed dispersal efficiency and accuracy. The algorithm's ability to optimize the process for different environmental conditions and seed types could have significant benefits for plant species survival and ecosystem biodiversity.



\end{document}




